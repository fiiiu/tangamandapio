\documentclass[12pt]{article}

\usepackage{cite}
\usepackage{graphicx}

\title{Conectarse a Snappy a través de la consola serial}

\begin{document}

Estoy trabajando en \href{https://developer.ubuntu.com/en/snappy/}
{Snappy Ubuntu} con un BeagleBone Black Rev C y un Raspberry Pi 2. Es posible
conectarse a los dispositivos a través de la consola serial para monitorear
errores durante el arranque y para controlar el sistema. Luego de mi primer
cable quemado, decidí escribir las instrucciones para conectar los cables y así
reducir la probabilidad de que mis dispositivos prendan fuego.

Lo primero, por supuesto, es
\href{https://developer.ubuntu.com/en/snappy/start/}
{instalar snappy en la tarjeta SD}. Luego, comprar uno de estos
\href{http://www.adafruit.com/products/954}{cables FTDI} que convierten de USB
a serial. Este cable se puede usar en ambos dispositivos porque tiene los
conectores separados. El cable \emph{negro} es \textbf{tierra} (GND), el cable
\emph{blanco} es para \textbf{recibir datos en el puerto USB}, el cable
\emph{verde} es para \textbf{transmitir datos desde el puerto USB} ye el cable
\emph{rojo} es de \textbf{corriente eléctrica de 5V}.

\section{BeagleBone Black}

El \href{https://www.adafruit.com/product/1876}{BeagleBone Black Rev C} tiene 6
pines. Hay un punto blanco al lado del primer pin. Conecte el cable negro GND
al primer pin, el cable verde al cuarto pin y el cable blanco al quinto pin.
En el BeagleBone\textbf{NUNCA conecte el cable rojo}.

\begin{center}
  \href{
    https://ia601501.us.archive.org/11/items/snappy-bbb/1-board-serial-pins-compress.jpg}{
    \includegraphics[width=500px]{
      https://ia601501.us.archive.org/11/items/snappy-bbb/1-board-serial-pins-compress.jpg}
  }
  \caption{BeagleBone con los pines seriales resaltados}
\end{center}

\begin{center}
  \href{
    https://ia601501.us.archive.org/11/items/snappy-bbb/2-1-serial_cables.jpg}{
    \includegraphics[width=500px]{
      https://ia601501.us.archive.org/11/items/snappy-bbb/2-1-serial_cables.jpg}
  }
  \capation{BeagleBone con los cables conectados}
\end{center}

\section{Raspberry Pi}

El \href{https://www.adafruit.com/products/2358}{Raspberry Pi 2 Model B} tiene
40 pines. \href{http://www.pighixxx.com/test/2015/06/raspberry-pi-v2-mod-b-pinout/#prettyPhoto[gallery1105]/0/}
{Aquí} hay un bonito dibujo del pinout con los números y las funciones de los
pines.

Conecte el cable negro GND al sexto pin, el cable blanco al octavo pin y el
cable verde al décimo pin. De forma opcional puede darle corriente al Raspberry
conectando el cable rojo al segundo pin, pero si hace esto NO conecte el cable
de poder USB porque \textbf{no es posible usar ambas fuentes de poder}.

\begin{center}
  \href{
    https://ia601502.us.archive.org/29/items/image20150609_054825862/6-pins-highlight.jpg}{
    \includegraphics[width=500px]{
      https://ia601502.us.archive.org/29/items/image20150609_054825862/6-pins-highlight.jpg}
  }
  \caption{Raspberry con los pines seriales resaltados}
\end{center}

\begin{center}
  \href{
    https://ia601502.us.archive.org/29/items/image20150609_054825862/7-serial.jpg}{
    \includegraphics[width=500px]{
      https://ia601502.us.archive.org/29/items/image20150609_054825862/7-serial.jpg}
  }
  \caption{Raspberry con los calbes conectados}
\end{center}

\section{Consola serial}

Instale screen:

\begin{verbatim}
  sudo apt-get install screen
\end{verbatim}

Conecte el otro extremo del cable a un puerto USB de su computadora y busque el
nombre del dispositivo tty:

\begin{verbatim}
  ls /dev/ttyUSB*
\end{verbatim}

Inicie la sesión de screen en la terminal serial, reemplazando
\verb$/dev/ttyUSB?$ con el número de su dispositivo.

\begin{verbatim}
  sudo screen {/dev/ttyUSB?} 115200
\end{verbatim}

Por último, inserte la tarjeta SD y conecte el cable de poder de su
dispositivo. Ahora podrá ver los mensajes de arranque en la termina, y al final
snappy le permitirá inciar la sesión. El usuario predeterminado es
\emph{ubuntu} y la contraseña predeterminada también es \emph{ubuntu}.

\begin{center}
  \href{
    https://ia601501.us.archive.org/11/items/snappy-bbb/6-booted.png}{
    \includegraphics[width=500px]{
      https://ia601501.us.archive.org/11/items/snappy-bbb/6-booted.png}
  }
  \caption{Consola serial iniciada}
\end{center}

Snappy tiene un servidor SSH preinstalado. Una vez que el arranque ha sido
exitoso, se puede iniciar sesión a través de SSH, lo que nos da una mejor
experiencia. Esta terminal serial se puede usar para obtener la dirección IP
del dispositivo:

\begin{verbatim}
  ip addr show
\end{verbatim}

Y desde una terminal diferente, reemplazando \verb$ip$ con la dirección IP del
dispositivo:

\begin{verbatim}
  ssh ubuntu@{ip}
\end{verbatim}

Ahora estamos listos para empezar a jugar con snappy en el dispositivo. No se
pierdan del
\href{https://developer.ubuntu.com/en/snappy/tutorials/using-snappy/}{tour}.

Para salir de la consola serial, presione \verb$CTRL+a$ y luego \verb$k$.

\subsection{Más información sobre los cables y la consola serial}

\begin{itemize}
  \item \url{http://elinux.org/Beagleboard:BeagleBone_Black_Serial}
  \item \url{https://learn.adafruit.com/adafruits-raspberry-pi-lesson-5-using-a-console-cable/connect-the-lead}
  \item \url{http://dave.cheney.net/2013/09/22/two-point-five-ways-to-access-the-serial-console-on-your-beaglebone-black}
\end{itemize}

\end{document}
