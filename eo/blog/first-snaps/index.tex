\documentclass[12pt]{article}

\usepackage{cite}
\usepackage{graphicx}

\title{
  On writing my first snaps, and keeping my passwords secure everywhere}

\begin{document}

I've been writing snaps for a year, more or less. But all the snaps I had
written were a one-time thing to reproduce a bug, automate a scenario, or
explore a specific code path. A few months ago I decided to write one snap
that I would use
day-to-day. The application I use the most by far is keepass, so that seemed
like a good place to start.

\section{kpcli}

With \href{https://github.com/alecsammon/kpcli}{kpcli} I access my password
database from the command line. It's just one big perl script, but not as ugly
as it sounds :) That was a great first experiment because I haven't tried perl
in a snap before, and it would let me access the passwords from my beagle
bone black in a secure and confined way.

Because it is a script, I just need to copy it into the snap. Of course,
there's a plugin for that! \href{http://snapcraft.io/}{snapcraft} is a
meta-packaging tool to generate snaps. It comes with plugins for many
packaging tools, \emph{copy} being the most simple of them. Run:

\begin{verbatim}
  $ sudo apt install snapcraft
  $ snapcraft help copy
\end{verbatim}

The next step is to look at the dependencies. Luckily, they are right there in
the \href{https://github.com/alecsammon/kpcli/blob/master/README.md}{README}
file, in the form of Ubuntu packages. snapcraft calls them
\emph{stage-packages}, and they will be installed into the snap during the
build.

Now I have almost all the difficult details to make the snap, and it was
surprisingly easy. With this information, I wrote a part
(\href{http://snapcraft.io/create/#parts-intro}{read more about parts}):

\begin{verbatim}
  kpcli:
    source: https://github.com/alecsammon/kpcli.git
    plugin: copy
    files:
      kpcli.pl: scripts/kpcli.pl
    stage-packages:
      - perl-base
      - libcrypt-rijndael-perl
      - libterm-readkey-perl
      - libsort-naturally-perl
      - libfile-keepass-perl
      - libterm-shellui-perl
      - libterm-readline-gnu-perl
      - libclone-perl
\end{verbatim}

I am just missing a way to tell the command where to find the perl libraries
that I am bundling inside the snap. There is
\href{https://bugs.launchpad.net/snapcraft/+bug/1583259}{work in progress} to
make it easier to set up environment variables before launching the command;
but for my first simple snap, quick and easy will do. I wrote a wrapper script:

\begin{verbatim}
  #!/bin/sh

  $SNAP/usr/bin/perl -I $SNAP/usr/share/perl/5.22 -I $SNAP/usr/lib/x86_64-linux-gnu/perl/5.22 -I $SNAP/usr/lib/x86_64-linux-gnu/perl5/5.22 -I $SNAP/usr/share/perl5 $SNAP/scripts/kpcli.pl
\end{verbatim}

and the corresponding part:

\begin{verbatim}
  launcher:
    plugin: copy
    files:
      run.sh: scripts/run.sh
\end{verbatim}

Finally, I added details about the \emph{name}, \emph{version}, \emph{summary}
and \emph{description} of the snap, and defined the kpcli \emph{app}. You can
\href{https://github.com/ubuntu/snappy-playpen/blob/master/kpcli/snapcraft.yaml}
{download the source of this snapcraft.yaml}, put it in a directory and run
\textbf{snapcraft} in there to generate it yourself. Or try mine installing it
from the store:

\begin{verbatim}
  $ sudo snap install kpcli-elopio
  $ kpcli-elopio.kpcli
\end{verbatim}

\section{KeepassX}

Ok, that solves command line access, but I also want some eye-candy, sometimes.
\href{https://github.com/keepassx/keepassx.git}{KeepassX} gives me access to
the same passwords database through a GUI. By the time I finished my CLI snap I
saw work from \href{https://kyrofa.com/}{kyrofa},
\href{http://blog.sergiusens.org}{sergiusens} and
\href{http://davidplanella.org/}{dpm} to make it easy to snap desktop apps and
use them in Ubuntu classic. I also was at a sprint sitting next to
\href{https://launchpad.net/~attente}{attente}, who knows everything I didn't
know about the desktop, so it was a great time to enter the world of
collaborative snapping.

Let me digress here for a moment to talk about collaboration.
\textbf{snapcraft} and \textbf{snapd} are free software, and a lot of their
design is based on the idea that the system can be extended and improved through
the shared work of many and diverse developers. \textbf{snapcraft} aims to be
the universal packaging tool, and \textbf{plugins} are the way we integrate the
wild variety of build systems into the lifecycle to generate a snap. KeepassX is
more complex than the CLI client, and it uses \emph{cmake} to generate the
executable. Well, guess what? There's a plugin for that! :)
(\href{https://github.com/snapcore/snapcraft/blob/master/snapcraft/plugins/cmake.py}
{see the source of the cmake plugin}). Also, this application uses the QT5 UI
libraries. Just like with the kpcli snap, that means I will require to set up
some environment variables and a launcher; but somebody else already did that and we
have a way to share and reuse \textbf{parts}. Run:

\begin{verbatim}
  $ snapcraft update
  $ snapcraft search qt5
  $ snapcraft define qt5conf
\end{verbatim}

There's a part for that!! And when this application is running, I would like it
to be integrated with \emph{unity7}, if available. That means I would need a few
extra permissions for my isolated app to access the unity7 services and
libraries installed in a classic Ubuntu system. I'm a lucky guy because there's
an interface for that!!!
(\href{https://github.com/snapcore/snapd/blob/master/interfaces/builtin/unity7.go}
{see the source of the unity7 interface}) By default, snaps will be fully
confined and won't be able to access anything they don't provide by themselves.
Through \textbf{interfaces}, we allow snaps to get extra permissions to consume
services, access devices and directories, and do anything that you could do in a
classic unconfined Linux system; but now in a controlled manner with the
explicit acknowledgement of the user for sensitive stuff. Run:

\begin{verbatim}
  $ snap interfaces
\end{verbatim}

We have plugins, parts and interfaces for the most common cases of the most
common apps. But the fun is just starting, and there are lots and lots of apps
out there that would benefit from the controlled dependencies, usable security
and transactional updates of snaps. We celebrate when we get new people into the
team, so if you want to participate and influence the future of this shiny
project, we can help you getting started
\href{https://developer.ubuntu.com/en/blog/2016/06/08/snappy-playpen-kickoff-highlights/}
{in the playpen}.

One last detail about collaboration is the relation with upstreams and
distribution maintainers. Snapping free software projects is a pleasure because
the source is available for us to inspect and decide on the best strategy to
build the snap. Then we spread the apps to more people, we get more feedback and
in many cases we end up sending patches to improve the upstream project. And I
already mentioned the use of Ubuntu packages during the build of the snaps. With
this we bootstrap from the awesome work that the \textbf{Debian} and
\textbf{Ubuntu} communities are doing, and we'll keep benefitting from both
sides as we need debs to build some snaps, and the two play nicely together.

Aaaand back on track, because with all the work shared by the community this
KeepassX snap is almost done by itself, but not quite.

I start by creating a new \emph{snapcraft.yaml} file an defining the snap
details:

\begin{verbatim}
  name: keepassx-elopio
  version: "2.0.2"
  summary: KeePassX is a cross platform password safe
  description: |
    KeePassX is an application for people with extremly high demands on secure
    personal data management. It has a light interface, is cross platform and
    published under the terms of the GNU General Public License.
\end{verbatim}

Next, the app:

\begin{verbatim}
  apps:
    keepassx:
      command: qt5-launch keepassx
      plugs: [unity7, opengl]
\end{verbatim}

Take a look at that qt5-launch, which is provided from the shared qt
\textbf{part} that I will reference in a few lines. \emph{keepassx} is a binary
that lives inside the snap, and that we will compile using a \emph{snapcraft}
\textbf{plugin}. Also look at those plugs. That's how we tell the system that a
snap wants to consume an \textbf{interface} slot of the same name.

Followed by the parts:

\begin{verbatim}
  parts:
    keepassx:
      source: https://github.com/keepassx/keepassx.git
      source-tag: 2.0.2
      plugin: cmake
      build-packages:
        - g++
        - qtbase5-dev
        - libqt4-dev
        - libqt5x11extras5-dev
        - qttools5-dev
        - qttools5-dev-tools
        - libgcrypt20-dev
        - zlib1g-dev
      stage-packages:
        - libgtk2.0-0
        - libqt5gui5
        - unity-gtk2-module
        - appmenu-qt5
        - dconf-gsettings-backend
      after: [qt5conf]
\end{verbatim}

Note the new \emph{build-packages} section. Those packages will not end up in
the snap, they are only used during the compilation so they will be installed
in the host where you are running \emph{snapcraft}. I already discussed about
the \emph{stage-packages}, nothing new to see there. Now the \emph{after}
keyword is where a lot of the collaboration magic happens. It means that my
\emph{keepassx} part depends on the \emph{qt5conf} part. But I don't define the
\emph{qt5conf} part in my yaml file, so snapcraft will bring it from the pool
of shared parts
(\href{https://github.com/dplanella/qt5conf}
{see the source of the qt5conf part}). I love this because I didn't have to
write any of the qt complications, and when a qt expert makes a change to
improve it, I just need to run snapcraft again and my snap will be better with
no effort. This is where the \emph{qt5-launch} I used in the apps section comes
from.

I finish my yaml adding a last part:

\begin{verbatim}
  ...
  parts:
    keepassx:
      ...
    glibcompiledassets:
      plugin: copy
      files:
        gschemas.compiled: usr/share/glib-2.0/schemas/gschemas.compiled
        giomodule.cache: usr/lib/x86_64-linux-gnu/gio/modules/giomodule.cache
\end{verbatim}

This is required to save the KeepassX settings. It's currently not a nice way
to include the compiled files in the snap, and we are discussing about how to
improve it, so I won't dig too much here. It shows how simple yet flexible snaps
are: you can put anything into the snap, it doesn't matter how you got it. But
it also shows that we have things to improve, so let me insist on my invitation
for you to join our project. If there's something you hate, this is the right
time to change it. If there's something you like, there's always room to improve
it.
\href{http://media-cache-ak0.pinimg.com/736x/d7/cf/6e/d7cf6e3ec423dd7a02cc70a0aba8d99d.jpg}
{We have cookies}.

Punto final, café con tamal. Give it a try:

\begin{verbatim}
  $ sudo snap install keepassx-elopio
  $ keepassx-elopio.keepassx
\end{verbatim}

\begin{center}
  \href{
    https://ia600208.us.archive.org/24/items/elopio-screenshots/EkrankopioDe2016-07-0111-48-23.png}{
    \includegraphics[width=500px]{
      https://ia600208.us.archive.org/24/items/elopio-screenshots/EkrankopioDe2016-07-0111-48-23.png}
  }
  \caption{KeepassX running from a snap}
\end{center}

\end{document}
