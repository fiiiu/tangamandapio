\documentclass[12pt]{article}

\usepackage{graphicx}

\title{Test a stable release update}

\begin{document}

Ubuntu has a six month cycle for releases. This means that we work for six
months updating software, adding new features and making sure that it all works
consistently together when it's integrated into the operating system, and then
we release it. Once we make a release, it is considered stable and then we
almost exclusively add to that release critical bug fixes and patches for security
vulnerabilities. The exceptions are just a few, and are for software that's
important for Ubuntu and that we want to keep up-to-date with the latest
features even in stable releases.

These bug fixes, security patches and exceptional new features require a lot of
testing, because right after they are published they will reach all the Ubuntu
users that are in the stable release. And we want the release to remain stable,
never ever introduce a regression that will make those users unhappy.

We call all of them
\href{https://wiki.ubuntu.com/StableReleaseUpdates}{Stable Release Updates},
and we test them in the \emph{proposed} pocket of the Ubuntu archive. This is
obviously not enabled by default, so the brave souls that want to help us
testing the changes in proposed need to enable it.

Before we go on, I would recommend to test SRUs
\href{http://elopio.net/blog/install-ubuntu-in-vm/}
     {in a virtual machine}.
Once you enable proposed following this guide you will get constant and untested
updates from many packages, and these updates will break parts of your system
every now and then. It's not likely to be critical, but it can bother you if it
happens on the machine you need to do your work, or other stuff. And if
somebody makes a mistake, you might need to reinstall the system.

You will also have to find a package that needs testing.
\href{http://snapcraft.io}{Snapcraft} is one of the few exceptions allowed to
land in a stable release every week. So lets use it as an example. Lets say you
want to help us testing the upcoming release of snapcraft in Ubuntu 16.04.

With your machine ready and before enabling proposed, install the version
already released of the package you want to test. This way you'll test later an
update to the newer version, just what a normal user would get once the update
is approved and lands in the archive. So in a terminal, write:

\begin{verbatim}
  $ sudo apt update
  $ sudo apt install snapcraft
\end{verbatim}

Or replace \emph{snapcraft} with whatever package you are testing. If you are
doing it just during the weekend after I am writing this, the released
version of snapcraft will be 2.23. You can check it with:

\begin{verbatim}
  $ dpkg -s snapcraft | grep Version
  Version: 2.23
\end{verbatim}

Now, to enable proposed, open the \emph{Ubuntu Software} application, and
select \emph{Software & Updates} from the menu in the top of the window.

\begin{center}
  \href{
    https://archive.org/download/elopio-screenshots2/sru/1-open_sources.png}{
    \includegraphics[width=500px]{
    https://archive.org/download/elopio-screenshots2/sru/1-open_sources.png}
  }
  \caption{Ubuntu Software application}
\end{center}

From the \emph{Software & Updates} window, select the \emph{Developer Options}
tab. And check the item that says \emph{Pre-released updates}.

\begin{center}
  \href{
    https://archive.org/download/elopio-screenshots2/sru/2-enable_proposed.png}{
    \includegraphics[width=500px]{
    https://archive.org/download/elopio-screenshots2/sru/2-enable_proposed.png}
  }
  \caption{Deverloper Options tab of the Software & Updates application}
\end{center}

This will prompt for your password. Enter it, and then click the
\emph{Close} button. You will be asked if you want to reload your sources, so
yes, click the \emph{Reload} button.

\begin{center}
  \href{
    https://archive.org/download/elopio-screenshots2/sru/3-reload.png}{
    \includegraphics[width=500px]{
    https://archive.org/download/elopio-screenshots2/sru/3-reload.png}
  }
  \caption{Reload sources}
\end{center}

Finally try to upgrade. If there is a newer version available in the proposed
pocket for the packet you are testing, now you will get it.

\begin{verbatim}
  $ sudo apt install snapcraft
  $ dpkg -s snapcraft | grep Version
  Version: 2.24
\end{verbatim}

Every time there is a proposed update, the package will have corresponding SRU
bugs with the tag "verification-needed". In the case of snapcraft this weekend,
this is the bug for the 2.24 proposed update:
\href{https://bugs.launchpad.net/ubuntu/+source/snapcraft/+bug/1650632}
     {https://bugs.launchpad.net/ubuntu/+source/snapcraft/+bug/1650632}

The SRU bugs will have instructions about how to test that the fix or the new
features released are correct. Follow those steps, and if you are confident
that they work as expected, change the tag to "verification-done". If you find
that the fix is not correct, change the tag to "verification-failed". In case
of doubt, you can leave a comment in the bug explaining what you tried and what
you found.

You can read more about SRUs in the
\href{https://wiki.ubuntu.com/StableReleaseUpdates}
     {Stable Release Updates wiki page},
and also in the
\href{https://wiki.ubuntu.com/QATeam/PerformingSRUVerification}
     {wiki page explaining how to perform verifications}.
This last page includes a section to find packages and bugs that need
verification. If you want to help the Ubuntu community, you can just jump in
and start verifying some of the pending bugs. It will be highly appreciated.

If you have questions or find a problem, join us in the
\href{https://rocket.ubuntu.com/channel/community}{Ubuntu Rocket Chat}.

\end{document}
