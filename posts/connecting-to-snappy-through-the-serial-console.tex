\documentclass[12pt]{article}

\usepackage{cite}
\usepackage{graphicx}

\title{Connecting to Snappy through the serial console }

\begin{document}

I'm working on \href{https://developer.ubuntu.com/en/snappy/}{Snappy Ubuntu}
with a BeagleBone Black Rev C and a Raspberry Pi 2. You can connect to the
boards through the serial console to watch for errors during the boot process
and to control the system. After my first burned cable I decided to write down
the instructions to connect the cables to reduce the likelihood of my boards
catching fire.

First, of course, you should
\href{https://developer.ubuntu.com/en/snappy/start/}{flash the SD card with
snappy}. Then, get one of these \href{http://www.adafruit.com/products/954}
{FTDI cables} that convert USB to serial. As this cable has the plugs
separated, it will work for both boads. The \emph{black} wire is
\textbf{ground} (GND), the \emph{white} wire is \textbf{receive into the USB port}, the \emph{green} wire is \textbf{transmit out of the USB port} and the
\emph{red} wire is \textbf{5V power}.

\section{BeagleBone Black}

The \href{https://www.adafruit.com/product/1876}{BeagleBone Black Rev C} has 6
pins. There's a white dot next to the first pin. Connect the black GND wire to
the first pin, the green wire to the fourth pin and the white wire to the fifth
pin. On the BeagleBone \textbf{NEVER connect the red power wire}.

\begin{center}
  \href{
    https://ia601501.us.archive.org/11/items/snappy-bbb/1-board-serial-pins-compress.jpg}{
    \includegraphics[width=500px]{
      https://ia601501.us.archive.org/11/items/snappy-bbb/1-board-serial-pins-compress.jpg}
  }
  \caption{BeagleBone with serial pins highlighted}
\end{center}

\begin{center}
  \href{
    https://ia601501.us.archive.org/11/items/snappy-bbb/2-1-serial_cables.jpg}{
    \includegraphics[width=500px]{
      https://ia601501.us.archive.org/11/items/snappy-bbb/2-1-serial_cables.jpg}
  }
  \capation{BeagleBone with cables connected}
\end{center}

\section{Raspberry Pi}

The \href{https://www.adafruit.com/products/2358}{Raspberry Pi 2 Model B} has
40 pins. \href{http://www.pighixxx.com/test/2015/06/raspberry-pi-v2-mod-b-pinout/#prettyPhoto[gallery1105]/0/}
{Here} you can find a nice pinout drawing with the numbers and functions of the
pins.

Connect the black GND wire to the sixth pin, the white wire to the eight pin
and the green wire to the tenth pin. You can optionally power the Raspberry
connecting the red wire to the second pin, but if you do this DO NOT connect
the USB power connector as you \textbf{can't have both power sources}.

\begin{center}
  \href{
    https://ia601502.us.archive.org/29/items/image20150609_054825862/6-pins-highlight.jpg}{
    \includegraphics[width=500px]{
      https://ia601502.us.archive.org/29/items/image20150609_054825862/6-pins-highlight.jpg}
  }
  \caption{Raspberry with serial pins highlighted}
\end{center}

\begin{center}
  \href{
    https://ia601502.us.archive.org/29/items/image20150609_054825862/7-serial.jpg}{
    \includegraphics[width=500px]{
      https://ia601502.us.archive.org/29/items/image20150609_054825862/7-serial.jpg}
  }
  \caption{Raspberry with cables connected}
\end{center}

\section{Serial console}

Install screen:

\begin{verbatim}
  sudo apt-get install screen
\end{verbatim}

Plug the other end of the cable to an USB port of your computer and figure out the name of the tty device:

\begin{verbatim}
  ls /dev/ttyUSB*
\end{verbatim}

And start the screen session on the serial terminal, replacing the
\verb$/dev/ttyUSB?$ with the number of your device.

\begin{verbatim}
  sudo screen {/dev/ttyUSB?} 115200
\end{verbatim}

Finally, insert the SD card and plug in the power source of the board. You
will see the boot messages on your terminal and in the end you will be
presented with the login prompt. The default user is \emph{ubuntu} and the
default password is \emph{ubuntu} too.

\begin{center}
  \href{
    https://ia601501.us.archive.org/11/items/snappy-bbb/6-booted.png}{
    \includegraphics[width=500px]{
      https://ia601501.us.archive.org/11/items/snappy-bbb/6-booted.png}
  }
  \caption{Booted serial console}
\end{center}

Snappy has an SSH server preinstalled, so once you booted successfully you can login through SSH, which will give you a better experience. You can use the serial console to query the IP address of the board:

\begin{verbatim}
  ip addr show
\end{verbatim}

And from a different terminal, replacing \verb$ip$ with the board IP address:

\begin{verbatim}
  ssh ubuntu@{ip}
\end{verbatim}

Now you are ready to start playing with snappy on the board. Don't miss the
\href{https://developer.ubuntu.com/en/snappy/tutorials/using-snappy/}{tour}.

To exit the serial console, press \verb$CTRL+a$ and then \verb$k$.

\subsection{More information about the cables and the serial console}

\begin{itemize}
  \item \url{http://elinux.org/Beagleboard:BeagleBone_Black_Serial}
  \item \url{https://learn.adafruit.com/adafruits-raspberry-pi-lesson-5-using-a-console-cable/connect-the-lead}
  \item \url{http://dave.cheney.net/2013/09/22/two-point-five-ways-to-access-the-serial-console-on-your-beaglebone-black}
\end{itemize}

\end{document}
